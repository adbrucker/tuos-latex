\documentclass[aspectratio=169]{tuos-presentation}
%\documentclass[compress,aspectratio=43]{tuos-presentation}
%\documentclass[aspectratio=169,colour]{tuos-presentation}
%\documentclass[aspectratio=169,compress,colour]{tuos-presentation}
%\documentclass[aspectratio=169,plainfonts,colour]{tuos-presentation}
\usepackage[T1]{fontenc}

\title{The University of Sheffield \LaTeX{} Template}
\subtitle{A Beamer-theme mimicking the PowerPoint guidelines of \\The
  University of Sheffield}
\institute[The University of Sheffield]
{Department of Computer Science, The University of Sheffield, Sheffield, UK}
  \author[A.D. Brucker] {%
    \href{http://www.brucker.uk/}{\textbf{Achim D. Brucker}}\\
    \texttt{\footnotesize\href{mailto:"Achim D. Brucker"
        <a.brucker@sheffield.ac.uk>}{a.brucker@sheffield.ac.uk}
      \hspace{.6cm}
      \url{http://www.brucker.uk/}}
    }

\begin{document}
\begin{frame}
  \maketitle
\end{frame}

\AgendaFrame

\section{Introduction}
\subsection{Introduction}
\begin{frame}
  \frametitle{A standard slide}
  \framesubtitle{With a frame subtitle}
  \begin{itemize}
  \item Only first word of slide title \ldots
    \begin{itemize}
    \item  And a second level 
      \begin{itemize}
      \item  Officially, a third level should be avoided
      \end{itemize}
    \end{itemize}
  \end{itemize}
  \[ x = \sum_{i=0}^{-\infty}\sqrt{-i}\]
\end{frame}

\section{Conclusion}
\subsection{Introduction}

\begin{frame}[plain]
  \frametitle{A plain slide}
  \framesubtitle{With a frame subtitle}
  \begin{itemize}
  \item Only first word of slide title \ldots
    \begin{itemize}
    \item  And a second level 
      \begin{itemize}
      \item  Officially, a third level should be avoided
      \end{itemize}
    \end{itemize}
  \end{itemize}
  \[ x = \sum_{i=0}^{-\infty}\sqrt{-i}\]
\end{frame}



\PartFrame{New Chapter}

\ThanksFrame

\CopyrightFrame

\end{document}
