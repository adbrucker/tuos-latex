\documentclass[a4paper]{tuos-letter}
\usepackage[english]{babel}

\LoadLetterOption{personal}

\begin{document}
\setkomavar{subject}{Guidance on letter layout}

\begin{letter}{
    Mr C Eague\\
    Job Title\\
    Company Name\\
    House Name/number, Street\\
    District\\
    Town/City\\
    Country\\
    Postcode
  }

  \opening{Dear Colin,}
  If a reference line is necessary, put it in bold, ranged left as in
  the example above.

  Do not change the size of the text as set in the template you
  download. All the body of the letter should be ranged left, ragged
  right, like this. Never indent the text, justify it or change the
  margins provided in the template. Text should be single line spaced,
  with two returns (one line space) between paragraphs.

  In the body of your letter, if you need to make text stand out put
  it in bold. For example, \textbf{please return this form by 1
    December 2008.} Never underline text.

  To quote a web address leave out the http:// part, remove the
  hyperlink (by using the Word Insert menu) and put the web address in
  bold with no underline and no full stop at the end, like this:
  \url{www.sheffield.ac.uk}

  If quoting the title of a publication, put it into italics like
  this: \textit{Home from Home.}

  There is useful guidance on the style, tone and content of letters
  at \url{www.sheffield.ac.uk/ marcoms/visualid/stationeryfaqs.htm}

  \closing{Yours sincerely}
\end{letter}
\end{document}
